\documentclass{amsart}
\usepackage[nohug,heads=littlevee]{diagrams}
\usepackage{mathrsfs}
\usepackage{amssymb}
\usepackage[latin1]{inputenc}
\usepackage{yfonts}
\usepackage{xr-hyper}
\usepackage[plainpages=false,pdfpagelabels]{hyperref}
\usepackage[nobysame,alphabetic]{amsrefs}
\usepackage{palatino}
\usepackage[left=2.2cm,top=3cm,right=2.5cm,bottom=1in,asymmetric]{geometry}
\usepackage{enumitem}


\setcounter{secnumdepth}{3}
\newcommand{\defnword}[1]{\textbf{#1}}
\newcommand{\comment}[1]{}

%theorems
\numberwithin{equation}{subsubsection}
\newtheorem{introthm}{Theorem}
\newtheorem{introcor}{Corollary}
\newtheorem*{introlem}{Lemma}
\newtheorem*{introprp}{Proposition}
\newtheorem{proposition}{Proposition}
\newtheorem{theorem}{Theorem}
\newtheorem{lemma}{Lemma}
\newtheorem{corollarylec}{Corollary}
\newtheorem{prp}[subsubsection]{Proposition}
\newtheorem{thm}[subsubsection]{Theorem}
\newtheorem*{thm*}{Theorem}
\newtheorem{lem}[subsubsection]{Lemma}
\newtheorem*{lem*}{Lemma}
\newtheorem{corollary}[subsubsection]{Corollary}
\theoremstyle{definition}
\newtheorem{defn}[subsubsection]{Definition}
\newtheorem{observation}{Observation}
\newtheorem{step}{Step}
\newtheorem{slogan}{Slogan}
\newtheorem{consequence}{Consequence}
\newtheorem{eg}[subsubsection]{Example}
\newtheorem*{claim}{Claim}
\newtheorem{claimn}[subsubsection]{Claim}
\newtheorem{definition}{Definition}
\newtheorem{question}{Question}
\newtheorem{caution}{Caution}
\newtheorem{notation}{Notation}
\newtheorem{fact}{Fact}
\theoremstyle{remark}
\newtheorem*{introrem}{Remark}
\newtheorem*{solution}{Solution}
\newtheorem{sdbar}[subsubsection]{Note on Notation}
\newtheorem{rem}[subsubsection]{Remark}
\newtheorem{assump}[subsubsection]{Assumption}
%\newtheorem{fact}[subsubsection]{Fact}
\newtheorem{listable}[subsubsection]{}
\newtheorem{example}{Example}
\newtheorem{remark}{Remark}

%diagrams
\newarrow{Equals}{=}{=}{}{=}{}
\newarrow{Implies}===={=>}
\newarrow{Iff}{<=}{=}{}{=}{=>}
% \makeatletter
% \newcommand\xleftrightarrow[2][]{\ext@arrow 0099{\longleftrightarrowfill@}{#1}{#2}}
% \def\longleftrightarrowfill@{\arrowfill@\leftarrow\relbar\rightarrow}
% \makeatother

%spaces
\newcommand{\Real}{\mathbb{R}}
\newcommand{\Int}{\mathbb{Z}}
\newcommand{\Nat}{\mathbb{N}}
\newcommand{\Comp}{\mathbb{C}}
\newcommand{\Field}{\mathbb{F}}
\newcommand{\Aff}{\mathbb{A}}
\newcommand{\Adele}{\mathbb{A}}
\newcommand{\Rat}{\mathbb{Q}}
\newcommand{\Lie}{\operatorname{Lie}}
\newcommand{\Dieu}{\mathbb{D}}
\newcommand{\g}{\mathfrak{g}}
\newcommand{\spf}{\mathfrak{sp}}
\newcommand{\rk}{\operatorname{rk}}
\newcommand{\trace}{\operatorname{trace}}

%category theory
\newcommand{\colim}{\operatorname{colim}}
\newcommand{\coker}{\operatorname{coker}}
\newcommand{\im}{\operatorname{im}}
\newcommand{\gr}{\operatorname{gr}}
\newcommand{\into}{\hookrightarrow}
\newcommand{\updot}{\bullet}
\newcommand{\downdot}{\bullet}
\newcommand{\chr}{\operatorname{char}}
\newcommand{\Vect}{\operatorname{Vect}}
\newcommand{\Obj}{\operatorname{Obj}}
\newcommand{\Kum}{\operatorname{Kum}}
\newcommand{\Art}{\operatorname{Art}}
\newcommand{\id}{\operatorname{id}}
\newcommand{\Set}{\operatorname{Set}}
\newcommand{\Def}{\operatorname{Def}}
\newcommand{\Inf}{\operatorname{Inf}}
\newcommand{\Pt}{\operatorname{Pt}}

%sheaves
\newcommand{\Rg}{\mathcal{O}}
\newcommand{\Reg}[1]{\Rg_{#1}}

%functors
\newcommand{\Hom}{\operatorname{Hom}}
\newcommand{\End}{\operatorname{End}}
\newcommand{\Tor}{\operatorname{Tor}}
\newcommand{\Ext}{\operatorname{Ext}}
\newcommand{\Aut}{\operatorname{Aut}}
\newcommand{\Isom}{\operatorname{Isom}}
\newcommand{\Sec}[2][\_\_]{\Gamma(#2,#1)}
\newcommand{\SHom}{\underline{\operatorname{Hom}}}
\newcommand{\SEnd}{\operatorname{\underline{End}}}
\newcommand{\SExt}{\underline{\operatorname{Ext}}}
\newcommand{\Der}{\operatorname{Der}}

%fieldtheory
\newcommand{\Gal}{\operatorname{Gal}}

%schemes
\newcommand{\Proj}{\operatorname{Proj}}
\newcommand{\codim}{\operatorname{codim}}
\newcommand{\Gm}{\mathbb{G}_m}
\newcommand{\Gmx}[1]{\hat{\mathbb{G}}_{m,#1}}

%p-adic hodge theory
\newcommand{\st}{\operatorname{st}}
\newcommand{\cris}{\operatorname{cris}}
\newcommand{\dR}{\operatorname{dR}}
\newcommand{\Bst}{B_{\st}}
\newcommand{\Bcris}{B_{\cris}}
\newcommand{\Bdr}{B_{\dR}}

%project
\newcommand{\Xet}[2][X]{{#1}_{\operatorname{et}}^{\operatorname{#2}}}
\newcommand{\Xfl}[2][X]{{#1}_{\operatorname{fl}}^{\operatorname{#2}}}
\newcommand{\Sig}{\mathfrak{S}}
\newcommand{\pow}[1]{[\vert#1\vert]}
\newcommand{\pdiv}[1]{\operatorname{pdiv}_{#1}}
\newcommand{\lpdiv}[1]{\operatorname{lpdiv}_{#1}}
\newcommand{\fin}[2]{\operatorname{fin}_{#1}^{#2}}
\newcommand{\Rep}{\operatorname{Rep}}
\newcommand{\res}{\operatorname{res}}
\newcommand{\MF}[2]{\operatorname{MF}^{#1}_{#2}}
\newcommand{\sM}[1]{\mathcal{M}_{[0,1]}^{\log}(#1)}
\newcommand{\spM}[1]{\mathcal{M}_{[0,1]}^{\log,\pol}(#1)}
\newcommand{\sMF}[1]{\mathcal{M}\mathcal{F}^{\log}_{[0,1]}(#1)}
\newcommand{\spMF}[1]{\mathcal{M}\mathcal{F}^{\log,\pol}_{[0,1]}(#1)}
\newcommand{\cM}[1]{\mathcal{M}_{[0,1]}(#1)}
\newcommand{\cMF}[1]{\mathcal{M}\mathcal{F}_{[0,1]}(#1)}
\newcommand{\cpMF}[1]{\mathcal{M}\mathcal{F}^{\pol}_{[0,1]}(#1)}
\newcommand{\Mod}[2][ ]{\operatorname{Mod}_{#2}^{#1}}
\newcommand{\BT}[2][ ]{\operatorname{BT}_{#2}^{#1}}
\newcommand{\Spec}{\operatorname{Spec}}
\newcommand{\Spf}{\operatorname{Spf}}
\newcommand{\lc}{\operatorname{LC}}
\newcommand{\Et}{\operatorname{Et}}
\newcommand{\et}{\operatorname{\acute{e}t}}
\newcommand{\ab}{\operatorname{ab}}
\newcommand{\sab}{\operatorname{sab}}
\newcommand{\mult}{\operatorname{mult}}
\newcommand{\unip}{\operatorname{unip}}
\newcommand{\cl}{\operatorname{cl}}
\newcommand{\wt}{\operatorname{wt}}
\newcommand{\dlog}{\operatorname{dlog}}
\newcommand{\dee}{\operatorname{d}}
\newcommand{\KS}{\operatorname{KS}}
\newcommand{\abs}[1]{\vert #1\vert}

%monoids
\newcommand{\Mon}{\operatorname{M}}
\newcommand{\gp}{\operatorname{gp}}
\newcommand{\sat}{\operatorname{sat}}
\newcommand{\tor}{\operatorname{tor}}
\newcommand{\intm}{\operatorname{int}}
\newcommand{\Fil}{\operatorname{Fil}}
\newcommand{\Fr}{\operatorname{Fr}}
\newcommand{\Ch}{\operatorname{Ch}}
\newcommand{\mx}{\mathfrak{m}}
\newcommand{\pf}{\mathfrak{p}}
\newcommand{\qf}{\mathfrak{q}}
\newcommand{\pol}{\operatorname{pol}}
\newcommand{\symm}{\operatorname{symm}}
\newcommand{\rank}{\operatorname{rank}}
\newcommand{\flogdiff}[1]{\hat{\Omega}^{1,\log}_{#1}}
\newcommand{\fdiff}[1]{\hat{\Omega}^1_{#1}}
\newcommand{\dual}[1]{{#1}^{\vee}}
\newcommand{\ip}[2]{\langle #1,#2\rangle}
\newcommand{\Sh}{\operatorname{Sh}}
\newcommand{\Ss}{\mathscr{S}}
\newcommand{\MT}{\operatorname{MT}}
\newcommand{\Res}{\operatorname{Res}}
\newcommand{\an}{\operatorname{an}}

%groups
\newcommand{\Ad}{\operatorname{Ad}}
\newcommand{\ad}{\operatorname{ad}}
\newcommand{\GSp}{\operatorname{GSp}}
\newcommand{\Sp}{\operatorname{Sp}}
\newcommand{\GL}{\operatorname{GL}}
\newcommand{\SL}{\operatorname{SL}}
\newcommand{\Orth}{\operatorname{O}}
\newcommand{\SO}{\operatorname{SO}}
\newcommand{\Unit}{\operatorname{U}}
\newcommand{\GU}{\operatorname{GU}}
\newcommand{\PGL}{\operatorname{PGL}}
\newcommand{\Spin}{\operatorname{Spin}}
\newcommand{\Gr}{\mathscr{G}}
\newcommand{\M}{\mathcal{M}}
\newcommand{\E}{\mathscr{E}}
\newcommand{\op}{\operatorname{op}}
\newcommand{\vect}[1]{\overrightarrow{\mathbf{#1}}} 

\title{Math 3311, Fall 2025: Homework 7}

\begin{document}
\maketitle

For a prime $p$, a \defnword{Sylow $p$-subgroup} of a finite group $G$ is a $p$-subgroup $P\leq G$ such that $p\nmid |G/P|$. We have already seen Sylow Theorem A, which basically tells us that Sylow $p$-subgroups always exist. Here are the remaining two:

\begin{introthm}
[Sylow theorem B]
Let $G$ be a finite group and let $\mathrm{Syl}_p(G)$ be the set of Sylow $p$-subgroups of $G$. Then $G$ acts \emph{transitively} on $\mathrm{Syl}_p(G)$ via conjugation.
\end{introthm}

\begin{introthm}
[Sylow theorem C]
Let $n_p$ be the number of Sylow $p$-subgroups of a finite group $G$. Then:
\begin{enumerate}
	\item $n_p$ divides $[G:P]$ for any Sylow $p$-subgroup $P\leq G$.
	\item $n_p\equiv 1 \pmod{p}$.
\end{enumerate}
\end{introthm}	

\begin{definition}
	A group $G$ is \defnword{simple} if it has no non-trivial, proper \emph{normal} subgroups.
\end{definition}


\begin{enumerate}[itemsep=0.15in]

\item Show that no group of order $100$ or $150$ is simple.

\emph{Hint: What can you say about the Sylow $5$-subgroups in both cases?}

Let $G$ be an arbitrary group of order 100. Let $P\in\mathrm{Syl}_5(G)$, which we know exists by Sylow A
By Sylow C, $|\mathrm{Syl}_5(G)|=n_5\mid[G:P]=100/25=4$ and $|\mathrm{Syl}_5(G)|=n_5\equiv1\pmod{5}$. The only $n_p$ satisfying this is $n_p=1$. So $P$ is a unique Sylow $p$-subgroup. Therefore $P$ is normal, and $G$ cannot be simple

Now, Let $H$ be an arbitrary group of order 150. Let $Q\in\mathrm{Syl}_5(G)$, which we know exists by Sylow A. Again, by Sylow C, $|\mathrm{Syl}_5(H)|=m_5\mid[G:Q]=150/25=6$ and $|\mathrm{Syl}_5(H)|=m_5\equiv1\pmod{5}$. The only $m_p$ satisfying this is $m_p=1$. So $Q$ is normal, and $H$ cannot be simple.

\vspace{0.2in}

\item Show that any simple finite \emph{abelian} group is isomorphic to $\Int/p\Int$ for some prime $p$.

\emph{Hint: Consider Sylow $p$-subgroups for various primes $p$. You might also want to use something from Homework 4.}



\vspace{0.2in}

\item Let $G$ be a group and $H\leq G$ be a subgroup. Consider the action of $H$ on the set of cosets $G/H$ via left multiplication. Show that we have
\[
	(G/H)^H = N_G(H)/H\subset G/H.
\]

\[
X^G = \{x\in X:\; g\cdot x = x\text{ for all $g\in G$}\}.
\]

$(G/H)^H=\{gH\in G/H:\;\forall g'\in G,g'\cdot gH=gH\}$





\vspace{0.2in}

\item Let $G$ be a finite group, $H\unlhd G$ a normal subgroup and $P\in \mathrm{Syl}_p(H)$ a Sylow $p$-subgroup of $H$. Show that $G = HN_G(P)$.

\emph{Hint: Note that, if $g\in G$, then $gPg^{-1}$ is also in $\mathrm{Syl}_p(H)$.}

\item Let $G$ be a finite group with $P\in \mathrm{Syl}_p(G)$. Show that, for any subgroup $M\leq G$ containing $N_G(P)$, we have $M = N_G(M)$. 

In other words, every subgroup containing $N_G(P)$ is its own normalizer.

\emph{Hint: Note that $M$ is normal in $N_G(M)$, and apply the previous problem.}

\vspace{0.2in}

\item Let $G$ be a simple group of order $60$, and let $n_2$ be the number of Sylow $2$-subgroups of $G$.
\begin{enumerate}
	\item Show that $n_2 = 5$ or $n_2 = 15$.
	\item If $n_2 = 15$, show that there exist two distinct Sylow $2$-subgroups $P_1$ and $P_2$ of $G$ such that 
\[
	|P_1\cap P_2| = 2.
\]
\end{enumerate}

\emph{Hint: For part (b), think about the possibilities for $n_5$, and what this says for how many elements can be contained in the union of all Sylow $2$-subgroups.}


\item Let $G$ be as in the previous problem. Suppose that one of the following is true:
\begin{enumerate}
	\item $n_2 = 5$;
	\item $G$ admits a subgroup $H\leq G$ of index $5$.
\end{enumerate}
Show that $G$ is isomorphic to a subgroup of $S_5$ of index $2$.


\item With the same notation as the previous two problems, suppose that $n_2 = 15$, and let $P_1,P_2\in \mathrm{Syl}_2(G)$ be two Sylow $2$-subgroups such that $|P_1\cap P_2| = 2$. Show that the normalizer $N_G(P_1\cap P_2)$ has index $5$ in $G$. Conclude that any simple group $G$ of order $60$ is isomorphic to a subgroup of $S_5$ of index $2$.

\emph{Hint: Note that both $P_1$ and $P_2$ are contained in $N_G(P_1\cap P_2)$.}


\item Suppose that we have $n,m\in \Int$ and consider the homomorphism
\[
\Int \xrightarrow{a\mapsto (a,a)}\Int/n\Int\times \Int/m\Int.
\]
\begin{enumerate}
	\item Show that the kernel of this homomorphism is generated by the least common multiple of $n$ and $m$.
	\item Show that the image of the homomorphism contains $d\Int/n\Int \times d\Int/m\Int$ where $d = \mathrm{gcd}(n,m)$.
	\item Conclude that, if $n$ and $m$ are relatively prime, there is a natural isomorphism of groups
\[
\Int/nm\Int \xrightarrow{\simeq}\Int/n\Int\times\Int/m\Int.
\]

\end{enumerate}

\end{enumerate}

\end{document}